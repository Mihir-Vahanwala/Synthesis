\documentclass{article}

\usepackage[parfill]{parskip}
\usepackage{amsmath}
\usepackage{amsthm}
\usepackage{amssymb}

\newtheorem{question}{Question}[section]
\newtheorem{example}{Example}[section]
\newtheorem{definition}{Definition}[section]

\newcommand{\bnalt}{\Bigm\vert}

\begin{document}

\section{Defining the problem}

\begin{question}[Adwait]
Given a PA formula $\phi(i, j, \dots)$ such that for all $i$, there exists a unique $j$ such that $\phi$ holds. Can we explicitly express $j$ in terms of $i$, given a proof of this dependence?
\end{question}

\begin{example}[Squaring in cyclic group]
Consider $$\phi \equiv (0 < i) \land (0 < j) \land (i < 9) \land (j < 9) \land (\exists q. i - 2j -9q = 0)$$
\end{example}
This is a \emph{permutation} over $\{1, \dots, 8\}$: we have that whenever $\phi(i, j)$ holds, $i \equiv 2j \mod 9$, $j \equiv 5i \mod 9$

A proof of uniqueness of $j$ for a given $i$ would proceed by contradiction, eliminating $i$ and then using the fact $9$ divides $2(j' - j)$ and that $2$ and $9$ are coprime, and that $-9 <  j - j' < 9$ 

To capture this example, our calculus somehow would need to support division with remainder, and even then, getting the explicit expression from a proof of uniqueness might be tricky...

\begin{example}[Chinese Remaindering]
Let
$$
\phi \equiv (0 < k) \land (k < 35) \land (\exists q_i. 5q_i + i = k) \land (\exists q_j . 7q_j + j = k)
$$
\end{example}
By Chinese remainder theorem, since $5$ and $7$ are coprime and $35 = 5 \cdot 7$, $k$ is uniquely determined by $i, j$: $k \equiv 21i + 15j \mod 35$

A clause that expresses coprimality of constants $c_1, c_2$ is 
$$
\exists q_1, q_2 . ~c_1q_1 + c_2 q_2 = 1
$$
It is from these $q_1, q_2$ that we can conclude about ``multiplicative inverses" modulo an integer, and use them in synthesis expressions that support modulo arithmetic.

In particular, $c_1 q_1 \equiv 1 \mod c_2$ and $0 \mod c_1$ and vice versa.

\textbf{From these two examples, it is evident our synthesis expressions must support modulo arithmetic.}

The following is a very elementary question
\begin{question}[Coprimality]
Let $c_1, c_2$ be arbitrary constants, and let $\psi$ be the sentence
$$
\exists q_1, q_2, r_1, r_2 .(0 < q_1, r_1 < c_2) \land (c_1q_1 - c_2q_2 = 1) \land (c_1r_1 - c_2r_2 = 0)
$$
In our logic, is $\neg \psi$ a theorem, or should it be an axiom?
\end{question}

\section{What can PA Synthesize?}
\begin{definition}[Well-behaved function, univariate]
A function $$f: \mathbb{Z} \rightarrow \mathbb{Z} \cup \{\bot\}$$ is said to be well behaved, if $\integers$ can be partitioned into finitely many intervals, such that for each interval $\interval$, there exists $c_\interval \in \mathbb{N}$, and functions $f_{\interval,0}, f_{\interval,1}, \dots, f_{\interval,c-1}$, which are either linear over $\mathbb{Z}$, or identically $\bot$, such that for all $x \in \interval$, $f(x) = f_{\interval,r}(x)$, where $x \equiv r \mod c_\interval$
\end{definition}

\begin{definition}[Implicitly defined, univariate]
Let $\phi(x, y)$ be a PA formula. We say that $\phi$ implicitly defines a function $f: \mathbb{Z} \rightarrow \mathbb{Z} \cup \{\bot\}$, if for all $x$, there is at most one $y$ for which $\phi(x, y)$ holds. $f(x) = y$, if there is a $y$ such that $\phi$ holds, and $f(x) = \bot$ otherwise.
\end{definition}

\begin{question}[Characterisation, univariate]
Does the following hold?

A function is well behaved if and only if it is implicitly defined by a PA formula.
\end{question}

It is straightforward to prove that a PA formula can be constructed from a well behaved function (hint: use divisibility predicate, the constant divisor is period $c$), but the other direction seems more profound.

\subsection{Proposed Normal Form for PA formulae}

In order to prove the harder direction, which seems intrinsically related to the very power of PA, the following grammar may be useful\footnotemark ($\alpha$ comes from the set of integers, $c$ from the set of common symbols, $x, y, z$ are variables):

\footnotetext{Why? I think because it is easy to synthesise a PA formula of this kind, given a well behaved function.}
$$
t ::= \alpha \bnalt c \bnalt x \bnalt \sum \alpha t 
$$
$$
p ::= (t = 0) \bnalt (t < 0) \bnalt \exists z. C
$$
$$
\ell ::= p \bnalt \neg p
$$
$$
C ::= \bigwedge \ell
$$
$$
F ::= \bigvee C
$$
Notice that there are many parallels that can be drawn with DNF, except that this \emph{isn't} DNF. Note that the analogue of literals is defined recursively, and hence, the conversion to this normal form also processes literals inductively, from innermost to outermost.

In this form, we insist that if $z$ is quantified, then every subformula in the scope of the quantifier \emph{must} contain $z$. This is not that principled an approach, and might be bad from implementation point of view, because we end up introducing \emph{more} quantifiers. But, the goal of this construction is to demonstrate equivalence.

Without loss of generality, we assume that for every quantification, a fresh variable is used. A way to convert any arbitrary formula to this normal form is:
\begin{enumerate}
\item Identify ``literals" $\ell$ as required: recursively replace $\neg\left( \bigwedge \psi \right)$ by $\bigvee \neg \psi$, $\neg\left( \bigvee \psi \right)$ by $\bigwedge\neg \psi$, $\neg \neg \psi$ by $\psi$, $\forall z. \psi$ by $\neg \exists z. \neg \psi$, $\neg \forall z. \psi$ by $\exists z. \neg \psi$

\item At this point, consider the innermost quantified formulae. They are of the form $\exists z. \psi$, where $\psi$ is in NNF, from the previous step. Convert this to DNF, and replace $\exists z. \bigvee_i \psi_i$ by $\bigvee_i \exists z_i. \psi_i$. If $\psi_i$ is independent of $z_i$, drop the quantifier. Finally, reconvert the decomposed formula back to NNF. (if there was a $\neg$ outside $\exists z. \psi$ to begin with, push it in)

\item Apply this procedure inductively to the outer quantifiers too. Here, the ``literals" also include processed quantified formulae from the previous step.

\item Having obtained all quantified subformulae in the right format, distribute, and convert the resulting ``NNF" formula into ``DNF"
\end{enumerate}
\end{document}