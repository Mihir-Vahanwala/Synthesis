\documentclass{article}

\usepackage[parfill]{parskip}
\usepackage{amsmath}
\usepackage{amsthm}
\usepackage{amssymb}

\newtheorem{question}{Question}[section]
\newtheorem{example}{Example}[section]
\newtheorem{definition}{Definition}[section]

\begin{document}

\section{Defining the problem}

\begin{question}[Adwait]
Given a PA formula $\phi(i, j, \dots)$ such that for all $i$, there exists a unique $j$ such that $\phi$ holds. Can we explicitly express $j$ in terms of $i$, given a proof of this dependence?
\end{question}

\begin{example}[Squaring in cyclic group]
Consider $$\phi \equiv (0 < i) \land (0 < j) \land (i < 9) \land (j < 9) \land (\exists q. i - 2j -9q = 0)$$
\end{example}
This is a \emph{permutation} over $\{1, \dots, 8\}$: we have that whenever $\phi(i, j)$ holds, $i \equiv 2j \mod 9$, $j \equiv 5i \mod 9$

A proof of uniqueness of $j$ for a given $i$ would proceed by contradiction, eliminating $i$ and then using the fact $9$ divides $2(j' - j)$ and that $2$ and $9$ are coprime, and that $-9 <  j - j' < 9$ 

To capture this example, our calculus somehow would need to support division with remainder, and even then, getting the explicit expression from a proof of uniqueness might be tricky...

\begin{example}[Chinese Remaindering]
Let
$$
\phi \equiv (0 < k) \land (k < 35) \land (\exists q_i. 5q_i + i = k) \land (\exists q_j . 7q_j + j = k)
$$
\end{example}
By Chinese remainder theorem, since $5$ and $7$ are coprime and $35 = 5 \cdot 7$, $k$ is uniquely determined by $i, j$: $k \equiv 21i + 15j \mod 35$

A clause that expresses coprimality of constants $c_1, c_2$ is 
$$
\exists q_1, q_2 . ~c_1q_1 + c_2 q_2 = 1
$$
It is from these $q_1, q_2$ that we can conclude about ``multiplicative inverses" modulo an integer, and use them in synthesis expressions that support modulo arithmetic.

In particular, $c_1 q_1 \equiv 1 \mod c_2$ and $0 \mod c_1$ and vice versa.

\textbf{From these two examples, it is evident our synthesis expressions must support modulo arithmetic.}

The following is a very elementary question
\begin{question}[Coprimality]
Let $c_1, c_2$ be arbitrary constants, and let $\psi$ be the sentence
$$
\exists q_1, q_2, r_1, r_2 .(0 < q_1, r_1 < c_2) \land (c_1q_1 - c_2q_2 = 1) \land (c_1r_1 - c_2r_2 = 0)
$$
In our logic, is $\neg \psi$ a theorem, or should it be an axiom?
\end{question}

\section{What can PA Synthesize?}
\begin{definition}[Well-behaved function, univariate]
A function $$f: \mathbb{Z} \rightarrow \mathbb{Z} \cup \{\bot\}$$ is said to be well behaved, if $\integers$ can be partitioned into finitely many intervals, such that for each interval $\interval$, there exists $c_\interval \in \mathbb{N}$, and functions $f_{\interval,0}, f_{\interval,1}, \dots, f_{\interval,c-1}$, which are either linear over $\mathbb{Z}$, or identically $\bot$, such that for all $x \in \interval$, $f(x) = f_{\interval,r}(x)$, where $x \equiv r \mod c_\interval$
\end{definition}

\begin{definition}[Implicitly defined, univariate]
Let $\phi(x, y)$ be a PA formula. We say that $\phi$ implicitly defines a function $f: \mathbb{Z} \rightarrow \mathbb{Z} \cup \{\bot\}$, if for all $x$, there is at most one $y$ for which $\phi(x, y)$ holds. $f(x) = y$, if there is a $y$ such that $\phi$ holds, and $f(x) = \bot$ otherwise.
\end{definition}

\begin{question}[Characterisation, univariate]
Does the following hold?

A function is well behaved if and only if it is implicitly defined by a PA formula.
\end{question}

It is straightforward to prove that a PA formula can be constructed from a well behaved function (hint: use divisibility predicate, the constant divisor is period $c$), but the other direction seems more profound.

\end{document}