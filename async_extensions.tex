\documentclass{article}


\usepackage[parfill]{parskip}
\usepackage{amsmath}
\usepackage{amsthm}
\usepackage{amssymb}
\usepackage{braket}
\usepackage{mathtools}
\usepackage{hyperref}
\usepackage{xcolor}
\usepackage{graphicx}
\usepackage{float}
\usepackage{bbm}
\usepackage{stmaryrd}

\newtheorem{question}{Question}[section]
\newtheorem{example}{Example}[section]
\newtheorem{definition}{Definition}[section]

\theoremstyle{remark}
\newtheorem{construction}{Construction}[section]

\DeclarePairedDelimiter{\ceil}{\lceil}{\rceil}
\DeclarePairedDelimiter{\floor}{\lfloor}{\rfloor}

\DeclarePairedDelimiter{\multiset}{\llbracket}{\rrbracket}

\DeclarePairedDelimiter{\transition}{[}{\rangle}

\DeclarePairedDelimiter{\pair}{\langle}{\rangle}

\renewcommand{\P}{\mathsf{P}}
\newcommand{\NP}{\mathsf{NP}}
\newcommand{\BQP}{\mathsf{BQP}}
\newcommand{\pspace}{\mathsf{PSPACE}}

\newcommand{\CFG}{\mathsf{CFG}}
\newcommand{\nonterminals}{\mathcal{X}}
\newcommand{\productions}{\mathcal{P}}

\newcommand{\program}{\mathfrak{P}}
\newcommand{\contexts}{\mathfrak{C}}

\newcommand{\multisets}{\mathbb{M}}

\newcommand{\naturals}{\mathbb{N}}
\newcommand{\integers}{\mathbb{Z}}

\newcommand{\Parikh}{\mathsf{Parikh}}

\newcommand{\PN}{\mathsf{PN}}

\newcommand{\schedule}[2]{\mathsf{schedule}^{#1}_{#2}}
\newcommand{\execute}[2]{\mathsf{execute}^{#1}_{#2}}
\newcommand{\suspend}[2]{\mathsf{suspend}^{#1}_{#2}}
\newcommand{\yield}[2]{\mathsf{yield}^{#1}_{#2}}

\newcommand{\widgetfam}{\mathcal{N}^{\clubsuit}}
\newcommand{\widgetnet}[1]{N_{#1}^{\clubsuit}}
\newcommand{\widgetS}[1]{S_{#1}^{\clubsuit}}
\newcommand{\widgetT}[1]{T_{#1}^{\clubsuit}}
\newcommand{\widgetF}[1]{F_{#1}^{\clubsuit}}

\begin{document}

\section{Extension 1: No Synchronous Function Calls, Unbounded Context Switching}
Denote the finite non-empty sets of global states by $D$ and handler names by $\Sigma$. In this case, an instance of the handler program is finite state, and the run of the program produces a regular language. However, a handler can get context switched out at any point, and hence, in the task buffer, we have to keep track of not just handler instances $\sigma$, but rather instruction labels $\lambda_\sigma^j$ to dispatch. 

Let $\Lambda$ be the set of all instruction labels across all handlers $\sigma$. Each handler also has a special label $\lambda^{end}_\sigma$.  Let $\Lambda^{end}$ be the set of such labels. Denote $\Lambda' = \Lambda \backslash \Lambda^{end}$. A handler is posted by adding $\lambda_\sigma^1$ to the task buffer.

The following can happen (note the non-deterministic choice) if the program is in global state $d$ and the control is at some instruction $\lambda$ in handler code.
\begin{itemize}
\item The instruction can be \textbf{executed}, thereby (non-deterministically) changing the global state to $d'$, sending the control to $\lambda'$, and posting $\lambda''$ to the task buffer. Note that posting is optional.
\item The handler instance is \textbf{suspended} at $\lambda_\sigma^j$ so that it can subsequently yield to the scheduler. In this case, the control jumps to $\lambda_\sigma^{end}$ and an instance of $\lambda_\sigma^j$ is posted to the task buffer.
\item If $\lambda = \lambda_\sigma^{end}$, then it can only \textbf{yield} to the scheduler.
\end{itemize}

\begin{construction}
This is captured by a Boolean Petri Net as follows
\begin{itemize}
\item The set of places $S = D \cup \Lambda' \cup (D \times \Lambda)$. $D$ captures global state as seen by the scheduler, $\Lambda'$ models the task buffer, $D \times \Lambda$ models the run of the handler in a scheduled context.
\item There is at most one marking in the states $D$. Initially, there is one marking in the starting state, and one marking in $\lambda^i_{\sigma}$ where $\sigma$ are the initial posts. To begin with, none of the places $(D \times \Lambda)$ are marked.
\item The set of transitions $T$ is the union of 
\begin{itemize}
\item $\{\schedule{\lambda}{d}\}_{(d, \lambda) \in D \times \Lambda'}$ in state $d$, schedule a handler that was at label $\lambda \in \Lambda'$
\item $\{\execute{\lambda}{d}\}_{(d, \lambda) \in D \times \Lambda'}$ in state $d$, execute the instruction at label $\lambda \in \Lambda'$. The next instruction $\lambda' \in \Lambda$, next state $d'$ and any posts $\lambda'' \in \Lambda'$ depend on the exact code.
\item  $\{\suspend{\lambda}{d}\}_{(d, \lambda) \in D \times \Lambda'}$ in state $d$, instead of executing the instruction at label $\lambda \in \Lambda'$, jump to the corresponding $\lambda^{end}\in \Lambda^{end}$, and post $\lambda$ to be scheduled later.
\item  $\{\yield{\lambda}{d}\}_{(d, \lambda) \in D \times \Lambda^{end}}$ return to the scheduler
\end{itemize}
\item These transitions are enabled and fired, according to $F$, given as:
\begin{itemize}
\item $F(\schedule{\lambda}{d}) = \pair{\multiset{d, \lambda}, \multiset{(d, \lambda)}}$
\item $F(\execute{\lambda}{d}) = \pair{\multiset{(d, \lambda)}, \multiset{\lambda'', (d', \lambda')}}$
\item $F(\suspend{\lambda}{d}) = \pair{\multiset{(d, \lambda)}, \multiset{\lambda, (d, \lambda^{end})}}$
\item $F(\yield{\lambda}{d}) = \pair{\multiset{(d, \lambda)}, \multiset{d}}$
\end{itemize}
\end{itemize}
\end{construction}

\begin{construction}
This is captured by a Vector Addition System with States (VASS) as follows
\begin{itemize}
\item The set $Q$ of states is the same as the set $D$ of global states.
\item The finite set of transitions $T \subseteq Q \times \integers^{|\Lambda|} \times Q$
\item The underlying vector models the instruction buffer. The set of transitions is constructed, taking into account what could happen for each instruction $\lambda$, from each state $q$.
\begin{itemize}
\item The start state is $q$
\item The end state $q'$ is the same as the state the instruction takes the program to
\item Let $t$ be the transition vector $\in \integers^{|\Lambda|}$. It has one entry, one for each instruction label. Let $\lambda'$ be the next instruction the program counter jumps to, and let $\lambda''$ be an instruction that is posted during the execution as an asynchronous call. We construct $t$ as follows:  
\begin{itemize}
\item For all $\lambda''' \in \Lambda$, $t(\lambda''') \gets 0$
\item $t(\lambda) \gets t(\lambda) - 1$
\item $ t(\lambda') \gets t(\lambda') + 1$
\item  $t(\lambda'') \gets t(\lambda'') + 1$
\end{itemize}
\end{itemize}
\end{itemize}
\end{construction}

\end{document}